\newpage
\section{Inne implementacje - Project Tango}
\paragraph{}
Kolejnym przykładem implementacji może być platforma Google Project Tango\footnote{https://get.google.com/tango/}. Jest to platforma rozszerzonej rzeczywistości zapoczątkowana przez Johny'ego Lee (współtwórcy między innymi Microsoft Kinect\footnote{https://en.wikipedia.org/wiki/Johnny\_Lee\_(computer\_scientist)}) w 2014. 
\paragraph{}
Idea projektu jest bardzo podobna jak przykład zaprezentowany w poprzednim rozdziale. Jednakże Project Tango to również podzespoły sprzętowe. Twórcy zastosowali specjalne kamery do pomiaru głębi oraz analizy ruchu (technologia podczerwieni). Kamery te korzystają z technologii Intel Real Sense. Dzięki temu urządzenie potrafi analizować obraz kamery i mapować go na trójwymiarowy obraz. Z dokładnością do milimetra urządzenie jest w stanie określić wymiary realnych elementów znajdujących się przed kamerą. Dzięki temu nie ma potrzeb używania zbędnych fizycznych markerów do określenia miejsca w którym znajduje się odbiorca z urządzeniem.
\paragraph{}
Firma Google zaprezentowała projekt w 2014 roku wraz z dwoma urządzeniami testowymi (The Yellowstone tablet,  The Peanut phone). Jednakze te urządzenia nie trafiły nigdy na rynek kompercyjny. Dopiero w 2016 roku firma Lenovo zaprezentowała pierwszy masowo produkowany telefon obsługujący Project Tango - Lenovo Phab2 Pro.
\paragraph{}
Projekt pod początku udostępnia developerom możliwość tworzenie aplikacji za pomocą API do języków Java oraz C. Dodatkowo udostępniona jest SDK (Software Development Kit) wraz z obszerną dokumentacją do platformy Unity \footnote{https://developers.google.com/tango/apis/unity/}.

\subsection{Wady i zalety}
\paragraph{}
{\color{red} opisać + dodać jakieś zdjęcie}

\paragraph{}
{\color{red}a moze dopisać porownanie z gearvr?}