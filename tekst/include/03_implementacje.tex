\chapter{Implementacje}
\paragraph{}
Rozdział prezentuje różne podejścia do stworzenia implementacji przykładowej sceny  w technologii rozszerzonej rzeczywistości.
Jako środowisko badawcze wybrano salę - labolatorium znajdującą się w Polsko-Japońskiej Akademii Technik Komputerowych. Salę tę wybrano, ponieważ znajduje się w podpiwniczeniu, co za tym idzie ilość światła dziennego jest znikoma. Pozwala to na bardzo łatwe stosowanie urządzeń projekcyjnych w ciągu dnia. Dodatkowo w sali znajduje się rura ciepłownicza poprowadzona po dwóch ścianach. Umiejscowienie tego elementu pozwala go wykorzystać do stworzenia wirtualnej sceny (np. animacja fal wody w środku rury).

\begin{center}
\includegraphics[width=0.9\textwidth]{images/s9.jpg}
\captionof{figure}{
Labolatorium PJATK
}
\end{center}

\paragraph{}
Głównym założeniem było stworzenie minigry opartej na grze z początku lat dziewięćdziesiątych o nazwie Lemmingi \cite{lemmings}. Pierwotnie gra została stworzona w 1991 roku na platformę Amiga.

Celem gry jest doprowadzenie grupy aktorów (tytułowych Lemmingów) do wyjścia (mety). Aktorzy automatycznie idą w jedną stronę. Każdemu z nich można włączyć jedną lub więcej umiejętność (np. umiejętność kopania, swobodnego spadania - spadochron). Aktorzy generowani są automatycznie w określonej sekwencji czasu.

\section{Implementacja w technologii hologramu}
\paragraph{}
Pierwotnym założeniem było stworzenie projektu za pomocą hologramu. Planowano wykorzystanie złudzenia optycznego stworzonego za pomocą światła na półprzezroczystej płaszczyźnie znajdującej się przed oczami odbiorcy. Analogiczną koncepcję prezentowało w przeszłości urządzenie Google Glass, a obecnie np. Microsoft Hololens lub Meta2.
\newline
Jednakże zamiast urządzenia nakładanego na głowę planowano użyć przezroczystą szybę znajdującą się na drzwiach wejściowych do labolatorium. Dzięki takiemu rozwiązaniu odbiór instalacji odbywałby się bez dodatkowych urządzeń, co za tym idzie interakcja byłaby bardziej naturalna.
Na szybie umieszczona zostałaby warstwa półprzezroczystej folii na której prezentowany byłby obraz za pomocą projektora multimedialnego ustawionego pod kątem około 30 stopni w górę w kierunku szyby. Projektor znajdowałby się na statywie w środku sali - technologia projekcja tylna.
\newline
Podczas eksperymentów próbowano wiele różnych folii oraz kilka ustawień projektorów. 
\newline
Zaobserwowano:
\begin{itemize}
	\item Umiejscowienie projektora pod kątem względem szyby powoduje bardziej naturalny obraz. Nie widać wtedy głównego strumienia światła z projektora (strumień światła nie jest prowadzony w linii prostej do odbiorcy), co za tym idzie  nie ma odczucia oślepienia.
    \item Użycie specjalnej folii do projekcji tylnej powoduje najlepszy odbiór. Inne folie przepuszczały zbyt małą ilość światła, bądź obraz z projektora był mało ostry.
    \item Użycie projektora w technologii LED okazało się lepsze, niż zastosowanie standardowego projektora multimedialnego. Mała odległość pomiędzy drzwiami, a urządzeniem pozwalała na użycie projektora z małą ilością lumenów.
\end{itemize}

\paragraph{}
Jednakże zrezygnowano z powyższego pomysłu, ponieważ napotkano problem nakładania obrazu z  elementami znajdującymi się w sali. Aby punkty wirtualne z ich rzeczywistymi odpowiednikami się nakładały i tworzyły spójny obraz (augumented reality) należałoby spoglądać przez szybę pod jednym wskazanym kątem. Co za tym idzie prawidłowy odbiór instalacji byłby zaburzony przez takie czynniki jak np. wzrost odbiorcy, kierunek wzroku czy też odległość od szyby. Urządzenia takie jak Microsoft Hololens niwelują ten problem poprzez stałe umiejscowienie półprzezroczystego ekranu w małej odległości od oka. 

\begin{center}
\includegraphics[width=0.9\textwidth]{images/hologramv1.png}
\captionof{figure}{
Poglądowy schemat działania hologramu
}
\end{center}

\begin{center}
\includegraphics[width=0.9\textwidth]{images/drzwi.jpg}
\captionof{figure}{
Próba implementacji w laboratorium
}
\end{center}

\section{Implementacja w technologii mappingu 3d}
\paragraph{}
Kolejnym pomysłem było zastosowanie video mappingu 3d. Jest to technologia często spotykana w branży rozrywkowej (jako tło sceny koncertowej) oraz do prezentowania przestrzeni architektonicznych (zarówno wewnętrznych jak i zewewnętrznych), czy też pojazdów jak i innych mniejszych przedmiotów. 
\newline
Technologia polega na oświetlaniu rzeczywistego elementu  źródłem światła z projektora multimedialnego. Najlepsze efekty można uzyskać w zaciemnionych pomieszczeniach oraz przy lampach z dużą ilością lumenów. Dzięki temu za pomocą kolorów można pokazywać lub uniewidoczniać elementy przy wykorzystaniu koloru czarnego. Podczas projekcji ciemnego koloru ilość światła z projektora jest bardzo znikoma, co za tym idzie powstaje złudzenie, że element oświetlony światłem czarnym jest niewidoczny.
\newline
Przy realizacji większości instalacji takiego typu stosowany jest wcześniej przygotowany obraz wideo. Instalacje nie są interaktywne. Jednakże dzięki temu nawet zaawansowane animacje trójwymiarowe obciążają sprzęt komputerowy tylko podczas renderowania (zamiany obiektów trójwymiarowych na strumień video). Wprowadzenie elementów interaktywnych (np. sterowanie grą) mocno obcąża kartę graficzną komputera.
\newline
Założono, że projekcja odbywać będzie się na dwóch prostopadłych do siebie ścianach. Takie podejście wymaga dwóch projektorów. Jednakże obraz z nich musi być synchronizowany, ponieważ elementy interaktywne (np. postacie) będą przemieszczały się z jednej ściany na drugą. Zastosowanie dwóch komputerów wymagałoby stworzenia protokołu komunikacyjnego. Założono, iż zastosowany będzie jeden komputer z wydajną kartą graficzną, która obsłuży dwa źródła wyjścia.
\begin{center}
\includegraphics[width=0.9\textwidth]{images/mappingv1.png}
\captionof{figure}{
Poglądowy schemat działania - mapping
}
\end{center}
\begin{center}
\includegraphics[width=0.9\textwidth]{images/implementacja.jpg}
\captionof{figure}{
Implementacja w laboratorium
}
\end{center}

Zaobserowano:
\begin{itemize}
	\item Efekty wizualne są odpowiednie, jednakże zauważalny jest brak głębi obrazu. Wszystkie elementy są dwuwymiarowe.
	\item Bardzo ważnym aspektem jest jakoś lampy projektora i jego umiejscowienie. Nawet drobne przesunięcie projektora w stosunku do ściany może zaburzyć odbiór instalacji.
\end{itemize}
