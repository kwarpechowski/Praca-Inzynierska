\newpage
\section{Wykorzystane technologie}

\subsection{Unity}
\paragraph{}
Unity jest obecnie najpopularniejszą platformą do tworzenia gier (zarówno trójwymiarowych jak i dwówymiarowych) na wiele platform sprzętowych. Silnik ten korzysta z API Direct3D (na urządzeniach Windows) oraz OpenGL.
\subsubsection{Dlaczego Unity}
\paragraph{}
Najnowsza wersja posiada natywne wsparcie do rozszerzonej oraz wirtualnej rzeczywistości. Narzędzie te posiada prosty, ergonomiczny interfejs co ułatwia pracę.
\paragraph{}
Bardzo pomocnym dodatkiem do narzędzia jest ,,Assets Store''. Jest to wirtualny sklep z komonentami do tworzenia gry. W projekcie zastosowałem tekstury i obiekty 3D pochodzące z tego źródła.

Dodatkowo silnik ten wspiera import modeli trójwymiarowych w bardzo dużej ilości specjalistycznych rozszerzeń plików.

\subsubsection{Alternatywne rozwiązania}
 \begin{tabular}{|c|c|c|}
 \hline
 \ & Unity & Unreal Engine \\ 
  \hline
 Wsparcie języków programowania & C\#, JavaScript, Boo & c++ \\  
  \hline
 Obsługa wielu ekranów & Tak & Nie \\
 \hline  
  Wsparcie dla Google Cardboard & Tak & Nie \\
  \hline   
  &  &  \\
  \hline   
  &  &  \\
  \hline   
\end{tabular}
\captionof{table}{Porównanie silników gier}

\subsubsection{Wybór języka programowania}
\paragraph{}
Środowisko Unity wspiera obsługę skryptów (animacje oraz logika biznesowa) w kilku językach programowania: C\#, UnityScript (zmodyfikowana wersja JavaScript)  oraz w przeszłości Boo. Podjęto decyzję, by w projekcie użyć język C\#, gdyż ów język jest najbardziej stabilny, posiada najbardziej rozbudowaną dokumentację oraz jest to najpopularniejszy język w specjalistycznej literaturze. Dodatkowym udogodnieniem  jest to, iż  język posiada wiele wbudowanych klas (np. do obsługi połączęń TCP) oraz niezliczoną ilość zewnętrznych bibliotek.
\subsubsection{Unity IDE}
\paragraph{}
Środowisko Unity jest multiplatformowe. Aplikacje można używać na dowolnym sytemie operacyjnym. Jednakże edycja skryptów odbywa się za pomocą zewnętrznego narzędzia. W systemie Mac OS X jest to MonoDevelop, natomiast w systemie Windows jest to VisualStudio w wersji Community. Opisywana aplikacja początkowo była tworzona na systemie Mac OS X, jednakże kołopoty ze środowiskiem MonoDevelop spowodwały decyzje o przeniesieniu środowiska na system Windows. Subiektywnie mogę stwierdzić, że stabilność oraz komfort pracy jest dużo lepszy w systemie Windows.
Dodatkową alternatywą dla MonoDevelop może okazać się Visual Studio Code. Jest to prosty multiplatformowy edytor posiadający obsługę języka C\# .

\subsubsection{Licencja i koszty}
\paragraph{}
Unity jest zamkniętym, licencjonowanym oprogramowaniem. Darmowa wersja (Personal Editon) pozwala na nielimitowane użycie, jednakże jest to okrojona edycja. Szersze informacje o ograniczeniach wersji Personal zawarte są w kolejnych rozdziałach. Licencja pozwala na komercyjne użycie przy limicie zarobków na poziomie stu tysięcy dolarów.
Komercyjna wersja (Professional Edition) jest płatna w modelu subskrybcyjnym (75 dolarów za miesiąc)\footnote{https://store.unity3d.com/subscribe}.
Na potrzeby opisywanego projektu zasotosowano Unity w wersji Personal Edition.

\subsection{Android}
\paragraph{}
Naturalnym wyborem technologii przy tworzeniu aplikacji na urządzenie sterujące byłoby Unity, gdyż te środowisko pozwala na kompilacje kodu na urządzenia mobilne(systemy: iOS, Android, Windows Phone, Tizen itp\footnote{https://unity3d.com/unity/multiplatform}). Jednakże Unity w wersji Personal Edition nie pozwala na uruchomienie warstwy sieciowej na urządzeniach mobilnych.
\paragraph{}
Na potrzeby implementacji przykładowego urządzenia sterującego wybrano platformę Android, gdyż ma ona największy udzła w rynku.\footnote{https://www.netmarketshare.com/operating-system-market-share.aspx?qprid=10\&qpcustomd=1} Proces tworzania aplikacji na tą platformę przebiega w języku Java.

\subsubsection{Android Studio}
\paragraph{}
opisac

\subsubsection{Zależności}
\begin{enumerate}
	\item Butter Knife
\end{enumerate}

\subsection{Komunikacja sieciowa}
\paragraph{}
Największym wyzwaniem było stworzenie dwukierunkowego protokołu komunikacyjnego pomiędzy serwerem (aplikacja napisana w środowisku Unity) oraz dowolnym kontrolerem lub w przyszłości innym urządzeniem wysłającym dane do aplikacji. Podstawowym założeniem było to iż, kontrolerem gry może być standardowy telefon komórkowy. Dodatkowo w przyszłości planowana jest rozbudowa o zdalne sterowanie za pomocą przeglądarki internetowej. Pierwotnie ozważane było użycie Bluetooth, jednak ograniczyłoby to zdalne sterowanie. Podjęto decyzję projektową o użyciu połączenia sieciowego. Rozważano następujące protokoły:

\subsubsection{UNET}
\paragraph{}
Unity wspiera natywną obsługę multiplayer - UNET, jednakże jest to zamknięty protokół. Komunikacja możliwa jest tylko pomiędzy aplikacjami stworzonymi w tym środowisku.

\subsubsection{HTTP (SOAP, REST)}
\paragraph{}
Komunikacja za pomocą HTTP (protokoły komunikacyjne takie jak np. SOAP, REST) są bardzo często spotykane. Jest to standard aplikacji internetowych. HTTP nadaje się do przesyłu dużych wolumenów danych, jednak niezbyt dobrze sprawdza się przy małych, lecz częstych połązeniach pomiędzy klientem, a serwerem. Duży narzut czasowy może spowodować transormacja danych do oraz z formatu JSON lub XML. Jednakże dużą zaletą wspomnianych protokołów jest prostota implementacji w większości języków programowania, gdyż są już gotowe komponenty.
\subsubsection{TCP}
\paragraph{}
{\color{red}Opisać, że TCP jest ogólnie lepsze - socket, ale to jest nadal połączeniowy, więc lepiej by było udp}
\subsubsection{UDP}
\paragraph{}
{\color{red}opisać, ze to najlepsze rozwiazanie - bezpolazeniowe}
