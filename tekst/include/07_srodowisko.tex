\section{Środowisko uruchomieniowe}
\paragraph{}
Powyżej opisana aplikacja została uruchomiona testowo w laboratorium na Polsko-Japońskiej Akademii Technik Komputerowych.

\subsection{Aplikacja główna - Unity}
\paragraph{}
Aplikacja została uruchomiona na komputerze przenośnym (laptop) posiadającym kartę graficzną umożliwiajacą podłączenie dwóch zewnętrznych ekranów - projektorów. Pierwszy z nich został połączony za pomocą złącza cyfrowego HDMI, natomiast drugi łączem DVI.
\subsection{Serwer komunikatów}
Serwer został uruchomiony na tym samym urządzeniu co aplikacja główna. Do uruchomienia nezbędne było zainstalowanie Node.JS wraz z menadżerem zależności - NPM.
\subsection{Aplikacja mobilna - kontroler}
\paragraph{}
Kontroler został uruchomiony na urządzeniu Xiaomi Mi4c wyposażonym w system Android w wersji 5.1. Urządzenie sprawdziło się jako kontroler, gdyż posiada ekran o rozmiarze 5 cali. Ilość pamięci operacyjnej była wystarczająca. Aplikacja zużywała tylko około 2-4\% pamięci RAM. 