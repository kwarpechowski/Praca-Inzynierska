\newpage
\section{Wnioski i dalszy rozwój}
\paragraph{}
Obserwując rynek nowych technologii można zauważyć duże zainteresowanie zarówno rozszerzoną jak i wirtualną rzeczywistością. Producenci eksperymentują z różnymi urządzeniami, jednakże większość ich jest w fazie testów i nie jest dostępna dla potencjalnych konsumentów.
Wiele z firm próbuje tworzyć własne standardy, jednakże żaden z nich nie zyskał jeszcze dużej popularności.
Wśród twórców oprogramowania interesujących się wirtualną rzeczywistością narasta trend stosowania oprogramowania Unity jako platofrmy do programowania. Trend ten wykorzystują przedsiębiorstwa produkujące urządzanie udostępniając specjalne Software Development Kit dla Unity. Warto wspomnieć, iż Microsoft planuje udostępnić know-how innym, mniejszym firmom\footnote{https://www.engadget.com/2016/06/01/microsoft-opens-the-hololens-platform-to-other-companies/}. Spowoduje to duży rozrost rynku.

\paragraph{}
Oczami programisty można stwierdzić, że stopa wejścia na nowy rynek jakim jest oprogramowanie urządzeń AR i VR jest dość prosty. Dzięki połączeniu kilku technologii w krótkim czasie można stworzyć nowoczesne narzędzie, które może zrewolucjonizować rynek.
