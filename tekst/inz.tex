\documentclass[inf, h]{pjatkThesis}
\usepackage[polish]{babel}

\usepackage[final]{graphicx}


\usepackage{url}
\usepackage{color}
\usepackage{lmodern}
\usepackage{caption}
\usepackage{listings}
\selectlanguage{polish}
\title{Platforma do tworzenia gier oraz animacji interaktywnych w środowisku rozszerzonej rzeczywistości}
\author{Kamil Warpechowski}

\linespread{1.5}


\fancypagestyle{firststyle}
{
   \fancyhf{}
   \fancyfoot[C]{
		Warszawa, \shortdate
   }
}

\usepackage{color}
\definecolor{bluekeywords}{rgb}{0.13,0.13,1}
\definecolor{greencomments}{rgb}{0,0.5,0}
\definecolor{redstrings}{rgb}{0.9,0,0}

\lstdefinelanguage{CSharp}
{
 morecomment = [l]{//}, 
 morecomment = [l]{///},
 morecomment = [s]{/*}{*/},
 morestring=[b]", 
 sensitive = true,
 morekeywords = {abstract,  event,  new,  struct,
   as,  explicit,  null,  switch,
   base,  extern,  object,  this,
   bool,  false,  operator,  throw,
   break,  finally,  out,  true,
   byte,  fixed,  override,  try,
   case,  float,  params,  typeof,
   catch,  for,  private,  uint,
   char,  foreach,  protected,  ulong,
   checked,  goto,  public,  unchecked,
   class,  if,  readonly,  unsafe,
   const,  implicit,  ref,  ushort,
   continue,  in,  return,  using,
   decimal,  int,  sbyte,  virtual,
   default,  interface,  sealed,  volatile,
   delegate,  internal,  short,  void,
   do,  is,  sizeof,  while,
   double,  lock,  stackalloc,   
   else,  long,  static,   
   enum,  namespace,  string}
}


\lstset{,
  showspaces=false,
  showtabs=false,
  breaklines=true,
  showstringspaces=false,
  breakatwhitespace=true,
  escapeinside={(*@}{@*)},
  frame=lrtb,
  aboveskip=2em
}

\setlength{\parindent}{1em}
\overfullrule=2cm
%\usepackage[T1]{fontenc}
%\usepackage[utf8]{inputenc}
\usepackage{times}
\usepackage{graphicx}

%\usepackage{fancyhdr}
\usepackage{makeidx}

\usepackage{lipsum}

\author{Kamil Warpechowski}
\album{s10709}
\title{Platforma do tworzenia gier oraz animacji interaktywnych w środowisku rozszerzonej rzeczywistości}
\type{Praca inżynierska}
\supervisor{dr inż. Michał Tomaszewski}
\location{Warszawa}
\date{wrzesień 2016}

\begin{document}

\maketitlepage

%\cleardoublepage
  
\pagestyle{fancyplain}

%%%  \include the `front matter'
\include{front}


%% create the table of contents
\cleardoublepage
\lhead[]{\fancyplain{}{\rightmark}}
\chead[\fancyplain{}{}]{\fancyplain{}{}}
\rhead[\fancyplain{}{\leftmark}]{\fancyplain{}{}}
\rhead[\fancyplain{}{}]{\fancyplain{}{}}
\lhead[\fancyplain{}{}]{\fancyplain{}{}}

\tableofcontents
%\listoffigures
%\listoftables

%\addcontentsline{toc}{chapter}{\listfigurename}
%\addcontentsline{toc}{chapter}{\listtablename}

%\cleardoublepage

%\newcommand{\publ}{}

\pagestyle{fancyplain}
\fancyhf{}
\fancyhead{} 
\renewcommand{\sectionmark}[1]{\markright{\it \thesection.\ #1}}
\renewcommand{\chaptermark}[1]{\markboth{\it \thechapter.\ #1}{}}


\lfoot[\fancyplain{}{}]                 {\fancyplain{}{}}
\cfoot[\fancyplain{\thepage}{\thepage}] {\fancyplain{\thepage}{\thepage}}
\rfoot[\fancyplain{}{}]                 {\fancyplain{}{}}

\renewcommand*\thesection{\thechapter.\arabic{section}}

\pagenumbering{arabic}
\baselineskip=22pt
%\cleardoublepage
\section{Wprowadzenie}
\paragraph{}
Ludzie od lat przyzwyczaili się korzystać z elektroniki oraz internetu na standardowych urządzeniach elektronicznych. Na początku lat dziewiędziesiątych do domów zaczęły trafiać komputery stacjonarne. Najpierw z modemami DSL\footnote{Digital Subscriber Line – technologia cyfrowego szerokopasmowego dostępu do internetu.[}, a następnie ze stałymi łączami światłowodowymi. Na przestrzeni lat korzystanie z ekranu w połączeniu z klawiaturą i myszą stało się dla ludzi naturalne.
\paragraph{}
Przez ostatnią dekadę na rynku pojawiły się interfejsy dotykowe. Popularność smartfonów, a następnie tabletów oraz urządzeń typu Wearables\footnote{https://pl.wikipedia.org/wiki/Wearables} spowodowało, że coraz bardziej sporadycznie korzystamy z standardowej fizycznej klawiatury.
\paragraph{}
Ekrany dotykowe pojawiły się nie tylko na urządzeniach telekomunikacyjnych, lecz także jako monitory w komputerach pokładowych samochodów oraz instalowane są w zagłówkach w samolotach jako multimedialne centrum rozrywki {\footnote{http://www.komputerswiat.pl/nowosci/wydarzenia/2012/28/boeing-z-androidem-na-pokladzie.aspx}.
\paragraph{}
Przez ostatnie kilka lat narasta trend poszukiwania innych metod dostępu do danych, zwłaszcza multimedialnych. Obecnie wiele przedsiębiorstw prowadzi badania nad nowymi, bardziej naturalnymi dla ludzi interfejsami, które nie wymagałyby użycia standardowych (sztucznych) urządzeń wejścia typu klawiatura czy myszka komputerowa.
\paragraph{}
Na przestrzeni lat interfejsy urządzeń elektronicznych ewoluowały pierwotnie z aplikacji sterowanych za pomocą wiersza polecań poprzez programy z graficznym interfejsem użytkownika (Graphic User Interface), aż do obecnie coraz bardziej popularnej i rozwijanej grupy interfejsów ,,naturalnych'' (Natural User Interface).
\begin{center}
\includegraphics[width=1\textwidth]{images/nui.png}
\captionof{figure}{
Ewolucja interfejsów użytkownika
}
\small {źródło: https://en.wikipedia.org/wiki/Natural\_user\_interface }
\end{center}
\paragraph{}
Wiele nowych urządzeń próbuje implementować sterowanie interfejsem użytkownika za pomocą gestów (np. Microsoft Kinect\footnote{http://www.xbox.com/pl-PL/xbox-one/accessories/kinect-for-xbox-one} lub sensor Leap Motion\footnote{https://www.leapmotion.com/}), czy też za pomocą myśli (np. Emotiv{\footnote{http://emotiv.com}). Na rynku widać duże zainteresowanie nową formą kontroli urządzeniami, zwłaszcza tymi bardziej naturalnymi dla człowieka. Jedankaże obecnie są to głównie eksperymenty nowej technologii. Nie ma na rynku obecnie wypracowanego popularnego standardu dostępu w grupie NUI.

\paragraph{}
W branży filmowej oraz gier wideo narasta trend używania nowych technologii do rozszerzania doznań jakie otrzymuje odbiorca.
W kinach odbywają są coraz częściej projekcje filmów stworzonych w technologii trójwymiarowej. Natomiast w ostatnim czasie pojawią się sale kinowe pozwalające na projekcję filmów trójwymiarowych wraz z dodatkowymi elementami takimi jak: drganie foteli, wiatr, dym, woda \footnote{http://cinema-city.pl/4dx-info}. Jednakże w obecnej chwili taki format rozrywki jest dość drogi, gdyż wymaga specjalnie przygotowanej sali kinowej oraz okularów, które pozwalają tworzyć iluzję przestrzenną. 

\subsection{Rozszerzona rzeczywistość}
\paragraph{}
Coraz bardziej popularne staje się pojęcie rozszerzonej rzeczywistości (ang. augmented reality). Jest to zbiór różnych technologii pozwalającej łączyć świat rzeczywisty z wirtualnym. Jest to jeszcze mało popularny sposób interakcji, lecz w ostatnim dziesięcioleciu rozwój (zarówno urządzeń jak i specjalistycznego oprogramowania) jest bardzo dynamiczny.
\paragraph{}
Pierwsze próby w tej dziedzinie odbywały się jeszcze w latach sześćdziesiątych amerykański naukowiec oraz artysta Myron Krueger prowadził badania nad wirtualną oraz rozszerzoną rzeczywistością. Jest on twórcą pojęcia środowiska responsywnego. ,,Jest to środowisko w którym działania użytkownika i odpowiada na nie w sposób przemyślany poprzez złożony system środków wizualnych i akustycznych, oraz dostosowuje się do powstałych w ten sposób nowych warunków środowiska.'' \footnote{http://www.techsty.art.pl/hipertekst/cyberprzestrzen/krueger.htm}. Stworzył on interaktywne instalacje takie jak Glowflow \footnote{http://dada.compart-bremen.de/item/artwork/1347}, Metaplay\footnote{http://dada.compart-bremen.de/item/artwork/1348} oraz Videoplace\footnote{http://dada.compart-bremen.de/item/artwork/1346}.

\begin{center}
\includegraphics[width=1\textwidth]{images/hololens.png}
\captionof{figure}{
Wizualizacja Microsoft HoloLens
}
\small {źródło: https://www.microsoft.com/microsoft-hololens/en-us/why-hololens }
\end{center}

\section*{Cel pracy}
\paragraph{}
Celem niniejszej pracy jest stworzenie platformy do tworzenia gier oraz interaktywnych animacji prezentowanej za pomocą rozszerzonej rzeczywistości sterowanej za pomocą zdalnego kontrolera. Praca zawiera przykłady różnych implementacji w technologiach z rodziny AR.
\paragraph{}
Przykłady zastosowanie zestawu aplikacji:
\begin{itemize}
	\item Prezentacje przestrzeni architektonicznych
	\item Rozrywka (np. gry)
	\item Reklama między innymi w miejscach użyteczności publicznych (np. centra handlowe)
\end{itemize}
\chapter{Implementacje}
\paragraph{}
Rozdział prezentuje różne podejścia do stworzenia implementacji przykładowej sceny  w technologii rozszerzonej rzeczywistości.
Jako środowisko badawcze wybrano salę - labolatorium znajdującą się w Polsko-Japońskiej Akademii Technik Komputerowych. Salę tę wybrano, ponieważ znajduje się w podpiwniczeniu, co za tym idzie ilość światła dziennego jest znikoma. Pozwala to na bardzo łatwe stosowanie urządzeń projekcyjnych w ciągu dnia. Dodatkowo w sali znajduje się rura ciepłownicza poprowadzona po dwóch ścianach. Umiejscowienie tego elementu pozwala go wykorzystać do stworzenia wirtualnej sceny (np. animacja fal wody w środku rury).

\begin{center}
\includegraphics[width=0.9\textwidth]{images/s9.jpg}
\captionof{figure}{
Labolatorium PJATK
}
\end{center}

\paragraph{}
Głównym założeniem było stworzenie minigry opartej na grze z początku lat dziewięćdziesiątych o nazwie Lemmingi \cite{lemmings}. Pierwotnie gra została stworzona w 1991 roku na platformę Amiga.

Celem gry jest doprowadzenie grupy aktorów (tytułowych Lemmingów) do wyjścia (mety). Aktorzy automatycznie idą w jedną stronę. Każdemu z nich można włączyć jedną lub więcej umiejętność (np. umiejętność kopania, swobodnego spadania - spadochron). Aktorzy generowani są automatycznie w określonej sekwencji czasu.

\section{Implementacja w technologii hologramu}
\paragraph{}
Pierwotnym założeniem było stworzenie projektu za pomocą hologramu. Planowano wykorzystanie złudzenia optycznego stworzonego za pomocą światła na półprzezroczystej płaszczyźnie znajdującej się przed oczami odbiorcy. Analogiczną koncepcję prezentowało w przeszłości urządzenie Google Glass, a obecnie np. Microsoft Hololens lub Meta2.
\newline
Jednakże zamiast urządzenia nakładanego na głowę planowano użyć przezroczystą szybę znajdującą się na drzwiach wejściowych do labolatorium. Dzięki takiemu rozwiązaniu odbiór instalacji odbywałby się bez dodatkowych urządzeń, co za tym idzie interakcja byłaby bardziej naturalna.
Na szybie umieszczona zostałaby warstwa półprzezroczystej folii na której prezentowany byłby obraz za pomocą projektora multimedialnego ustawionego pod kątem około 30 stopni w górę w kierunku szyby. Projektor znajdowałby się na statywie w środku sali - technologia projekcja tylna.
\newline
Podczas eksperymentów próbowano wiele różnych folii oraz kilka ustawień projektorów. 
\newline
Zaobserwowano:
\begin{itemize}
	\item Umiejscowienie projektora pod kątem względem szyby powoduje bardziej naturalny obraz. Nie widać wtedy głównego strumienia światła z projektora (strumień światła nie jest prowadzony w linii prostej do odbiorcy), co za tym idzie  nie ma odczucia oślepienia.
    \item Użycie specjalnej folii do projekcji tylnej powoduje najlepszy odbiór. Inne folie przepuszczały zbyt małą ilość światła, bądź obraz z projektora był mało ostry.
    \item Użycie projektora w technologii LED okazało się lepsze, niż zastosowanie standardowego projektora multimedialnego. Mała odległość pomiędzy drzwiami, a urządzeniem pozwalała na użycie projektora z małą ilością lumenów.
\end{itemize}

\paragraph{}
Jednakże zrezygnowano z powyższego pomysłu, ponieważ napotkano problem nakładania obrazu z  elementami znajdującymi się w sali. Aby punkty wirtualne z ich rzeczywistymi odpowiednikami się nakładały i tworzyły spójny obraz (augumented reality) należałoby spoglądać przez szybę pod jednym wskazanym kątem. Co za tym idzie prawidłowy odbiór instalacji byłby zaburzony przez takie czynniki jak np. wzrost odbiorcy, kierunek wzroku czy też odległość od szyby. Urządzenia takie jak Microsoft Hololens niwelują ten problem poprzez stałe umiejscowienie półprzezroczystego ekranu w małej odległości od oka. 

\begin{center}
\includegraphics[width=0.9\textwidth]{images/hologramv1.png}
\captionof{figure}{
Poglądowy schemat działania hologramu
}
\end{center}

\begin{center}
\includegraphics[width=0.9\textwidth]{images/drzwi.jpg}
\captionof{figure}{
Próba implementacji w laboratorium
}
\end{center}

\section{Implementacja w technologii mappingu 3d}
\paragraph{}
Kolejnym pomysłem było zastosowanie video mappingu 3d. Jest to technologia często spotykana w branży rozrywkowej (jako tło sceny koncertowej) oraz do prezentowania przestrzeni architektonicznych (zarówno wewnętrznych jak i zewewnętrznych), czy też pojazdów jak i innych mniejszych przedmiotów. 
\newline
Technologia polega na oświetlaniu rzeczywistego elementu  źródłem światła z projektora multimedialnego. Najlepsze efekty można uzyskać w zaciemnionych pomieszczeniach oraz przy lampach z dużą ilością lumenów. Dzięki temu za pomocą kolorów można pokazywać lub uniewidoczniać elementy przy wykorzystaniu koloru czarnego. Podczas projekcji ciemnego koloru ilość światła z projektora jest bardzo znikoma, co za tym idzie powstaje złudzenie, że element oświetlony światłem czarnym jest niewidoczny.
\newline
Przy realizacji większości instalacji takiego typu stosowany jest wcześniej przygotowany obraz wideo. Instalacje nie są interaktywne. Jednakże dzięki temu nawet zaawansowane animacje trójwymiarowe obciążają sprzęt komputerowy tylko podczas renderowania (zamiany obiektów trójwymiarowych na strumień video). Wprowadzenie elementów interaktywnych (np. sterowanie grą) mocno obcąża kartę graficzną komputera.
\newline
Założono, że projekcja odbywać będzie się na dwóch prostopadłych do siebie ścianach. Takie podejście wymaga dwóch projektorów. Jednakże obraz z nich musi być synchronizowany, ponieważ elementy interaktywne (np. postacie) będą przemieszczały się z jednej ściany na drugą. Zastosowanie dwóch komputerów wymagałoby stworzenia protokołu komunikacyjnego. Założono, iż zastosowany będzie jeden komputer z wydajną kartą graficzną, która obsłuży dwa źródła wyjścia.
\begin{center}
\includegraphics[width=0.9\textwidth]{images/mappingv1.png}
\captionof{figure}{
Poglądowy schemat działania - mapping
}
\end{center}
\begin{center}
\includegraphics[width=0.9\textwidth]{images/implementacja.jpg}
\captionof{figure}{
Implementacja w laboratorium
}
\end{center}

Zaobserowano:
\begin{itemize}
	\item Efekty wizualne są odpowiednie, jednakże zauważalny jest brak głębi obrazu. Wszystkie elementy są dwuwymiarowe.
	\item Bardzo ważnym aspektem jest jakoś lampy projektora i jego umiejscowienie. Nawet drobne przesunięcie projektora w stosunku do ściany może zaburzyć odbiór instalacji.
\end{itemize}

\newpage
\section{Wykorzystane technologie}

\subsection{Unity}
\paragraph{}
Unity jest obecnie najpopularniejszą platformą do tworzenia gier (zarówno trójwymiarowych jak i dwówymiarowych) na wiele platform sprzętowych. Silnik ten korzysta z API Direct3D (na urządzeniach Windows) oraz OpenGL.
\subsubsection{Dlaczego Unity}
\paragraph{}
Najnowsza wersja posiada natywne wsparcie do rozszerzonej oraz wirtualnej rzeczywistości. Narzędzie te posiada prosty, ergonomiczny interfejs co ułatwia pracę.
\paragraph{}
Bardzo pomocnym dodatkiem do narzędzia jest ,,Assets Store''. Jest to wirtualny sklep z komonentami do tworzenia gry. W projekcie zastosowałem tekstury i obiekty 3D pochodzące z tego źródła.

Dodatkowo silnik ten wspiera import modeli trójwymiarowych w bardzo dużej ilości specjalistycznych rozszerzeń plików.

\subsubsection{Alternatywne rozwiązania}
 \begin{tabular}{|c|c|c|}
 \hline
 \ & Unity & Unreal Engine \\ 
  \hline
 Wsparcie języków programowania & C\#, JavaScript, Boo & c++ \\  
  \hline
 Obsługa wielu ekranów & Tak & Nie \\
 \hline  
  Wsparcie dla Google Cardboard & Tak & Nie \\
  \hline   
  &  &  \\
  \hline   
  &  &  \\
  \hline   
\end{tabular}
\captionof{table}{Porównanie silników gier}

\subsubsection{Wybór języka programowania}
\paragraph{}
Środowisko Unity wspiera obsługę skryptów (animacje oraz logika biznesowa) w kilku językach programowania: C\#, UnityScript (zmodyfikowana wersja JavaScript)  oraz w przeszłości Boo. Podjęto decyzję, by w projekcie użyć język C\#, gdyż ów język jest najbardziej stabilny, posiada najbardziej rozbudowaną dokumentację oraz jest to najpopularniejszy język w specjalistycznej literaturze. Dodatkowym udogodnieniem  jest to, iż  język posiada wiele wbudowanych klas (np. do obsługi połączęń TCP) oraz niezliczoną ilość zewnętrznych bibliotek.
\subsubsection{Unity IDE}
\paragraph{}
Środowisko Unity jest multiplatformowe. Aplikacje można używać na dowolnym sytemie operacyjnym. Jednakże edycja skryptów odbywa się za pomocą zewnętrznego narzędzia. W systemie Mac OS X jest to MonoDevelop, natomiast w systemie Windows jest to VisualStudio w wersji Community. Opisywana aplikacja początkowo była tworzona na systemie Mac OS X, jednakże kołopoty ze środowiskiem MonoDevelop spowodwały decyzje o przeniesieniu środowiska na system Windows. Subiektywnie mogę stwierdzić, że stabilność oraz komfort pracy jest dużo lepszy w systemie Windows.
Dodatkową alternatywą dla MonoDevelop może okazać się Visual Studio Code. Jest to prosty multiplatformowy edytor posiadający obsługę języka C\# .

\subsubsection{Licencja i koszty}
\paragraph{}
Unity jest zamkniętym, licencjonowanym oprogramowaniem. Darmowa wersja (Personal Editon) pozwala na nielimitowane użycie, jednakże jest to okrojona edycja. Szersze informacje o ograniczeniach wersji Personal zawarte są w kolejnych rozdziałach. Licencja pozwala na komercyjne użycie przy limicie zarobków na poziomie stu tysięcy dolarów.
Komercyjna wersja (Professional Edition) jest płatna w modelu subskrybcyjnym (75 dolarów za miesiąc)\footnote{https://store.unity3d.com/subscribe}.
Na potrzeby opisywanego projektu zasotosowano Unity w wersji Personal Edition.

\subsection{Android}
\paragraph{}
Naturalnym wyborem technologii przy tworzeniu aplikacji na urządzenie sterujące byłoby Unity, gdyż te środowisko pozwala na kompilacje kodu na urządzenia mobilne(systemy: iOS, Android, Windows Phone, Tizen itp\footnote{https://unity3d.com/unity/multiplatform}). Jednakże Unity w wersji Personal Edition nie pozwala na uruchomienie warstwy sieciowej na urządzeniach mobilnych.
\paragraph{}
Na potrzeby implementacji przykładowego urządzenia sterującego wybrano platformę Android, gdyż ma ona największy udzła w rynku.\footnote{https://www.netmarketshare.com/operating-system-market-share.aspx?qprid=10\&qpcustomd=1} Proces tworzania aplikacji na tą platformę przebiega w języku Java.

\subsubsection{Android Studio}
\paragraph{}
opisac

\subsubsection{Zależności}
\begin{enumerate}
	\item Butter Knife
\end{enumerate}

\subsection{Komunikacja sieciowa}
\paragraph{}
Największym wyzwaniem było stworzenie dwukierunkowego protokołu komunikacyjnego pomiędzy serwerem (aplikacja napisana w środowisku Unity) oraz dowolnym kontrolerem lub w przyszłości innym urządzeniem wysłającym dane do aplikacji. Podstawowym założeniem było to iż, kontrolerem gry może być standardowy telefon komórkowy. Dodatkowo w przyszłości planowana jest rozbudowa o zdalne sterowanie za pomocą przeglądarki internetowej. Pierwotnie ozważane było użycie Bluetooth, jednak ograniczyłoby to zdalne sterowanie. Podjęto decyzję projektową o użyciu połączenia sieciowego. Rozważano następujące protokoły:

\subsubsection{UNET}
\paragraph{}
Unity wspiera natywną obsługę multiplayer - UNET, jednakże jest to zamknięty protokół. Komunikacja możliwa jest tylko pomiędzy aplikacjami stworzonymi w tym środowisku.

\subsubsection{HTTP (SOAP, REST)}
\paragraph{}
Komunikacja za pomocą HTTP (protokoły komunikacyjne takie jak np. SOAP, REST) są bardzo często spotykane. Jest to standard aplikacji internetowych. HTTP nadaje się do przesyłu dużych wolumenów danych, jednak niezbyt dobrze sprawdza się przy małych, lecz częstych połązeniach pomiędzy klientem, a serwerem. Duży narzut czasowy może spowodować transormacja danych do oraz z formatu JSON lub XML. Jednakże dużą zaletą wspomnianych protokołów jest prostota implementacji w większości języków programowania, gdyż są już gotowe komponenty.
\subsubsection{TCP}
\paragraph{}
{\color{red}Opisać, że TCP jest ogólnie lepsze - socket, ale to jest nadal połączeniowy, więc lepiej by było udp}
\subsubsection{UDP}
\paragraph{}
{\color{red}opisać, ze to najlepsze rozwiazanie - bezpolazeniowe}

\newpage
\section{Aplikacja główna}
\paragraph{}
Glownym zalozeniem aplikacji w Unity jest stworzenie wirtualnego świata składającego się z wizualizacji dwoch scian oraz elementow znajdujacych sie na nich. Po wyznaczonych elementach poruszać się beda aktorzy, czyli postacie gry.

{\color{red}tutaj będą makiety}

\subsection{Świat gry}
\paragraph{}
{\color{red}Opisać}

\begin{center}
\includegraphics[width=0.9\textwidth]{images/swiatgry.png}
\captionof{figure}{
Model w środowisku Unity
}
\small {źródło: własne }
\end{center}


\subsection{Aktorzy}
\paragraph{}
{\color{red}Opisać}


\begin{center}

 \begin{tabular}{|c|c|}
 \hline  
  &   \\
  \hline   
  &   \\
  \hline   
\end{tabular}
\captionof{table}{Właściwości aktora}
\paragraph{}
{\color{red}Opisać}
\end{center}

\begin{lstlisting}[language=CSharp]
public void SendInfo() {
	Network.SendMessage("hasax_"+this.hasAx);
	Network.SendMessage("hassh_"+this.hasSh);
	Network.SendMessage("hasdrabina_"+this.drabina);
	Network.SendMessage("isMove_"+this.isMove);
}
\end{lstlisting}
\captionof{lstlisting}{
	Do Poprawy
}

\subsection{Prefabrykaty}
\paragraph{}
W Unity możliwe jest używanie prefabrykatów (Prefabs \footnote{http://docs.unity3d.com/Manual/Prefabs.html}). Są obiekty lub grupy obiektów, które służą do wielokrotnego wykorzystywania. W Projekcie założono, że wszystkie reużwalne komponety (dziedziczone pomiędzy scenami) będą prefabrykatami.

Dodatkowo aktorzy gry (generowane dynamicznie) są również prefabrykatami. Instancje aktora są tworzone podczas działania aplikacji.

\subsubsection{Kamery}
\paragraph{}
Ważnym elementem gry są wirtualne kamery. To z nich renderowane jest ujęcie, czyli obraz gry. W projekcie zastosowano dwie kamery. Podczas konfiguracji uruchomieniowej należy ustawić by każda z kamer była wyświetlana na oddzielnym źródle obrazu (monitor, projektor). Dzięki temu projekt można uruchomić na dwóch prostopadłych ścianach.
\paragraph{}
Jako tło sceny wybrano jednolity kolor czarny, ponieważ ten kolor nie jest prezentowany podczas projekcji. Światłó z projektora jest w tym miejscu znikome, wręcz niewidoczne. Stosując taki prosty zabieg można łączyć elementy rzeczywiste (np. rura czy inne elementy stałe znajdujące się w labolatorium) z wirtualną rzeczywistością.

\begin{center}
\includegraphics[width=0.9\textwidth]{images/kamera1.png}
\includegraphics[width=0.9\textwidth]{images/kamera2.png}
\captionof{figure}{
Ujęcie wirtualnych kamer
}
\small {źródło: własne }
\end{center}

\paragraph{}
Środowisko Unity domyślnie nie ma włączonej opcji wspierania wielu kamer jednocześnie. Do opisywanego prefakbrykatu należy dodać abstrakcyjny GameObject z poniższym prostym skryptem, który przy uruchomieniu skopilowanej gry sprawdza dostępność sprzętową ekranów.

\begin{lstlisting}[language=CSharp]
using UnityEngine;
using System.Collections;

public class DisplayScript : MonoBehaviour
{
	void Start()
	{
		Debug.Log("displays connected: " + Display.displays.Length);
		if (Display.displays.Length > 1)
			Display.displays[1].Activate();
		if (Display.displays.Length > 2)
			Display.displays[2].Activate();
	}
}
\end{lstlisting}
\captionof{lstlisting}{
	Prosta aktywacja ekranów
}

\paragraph{}
Podczas testów uruchomieniowych przy dwóch kamerach występował problem z wydajnością karty graficznej. Zwłaszcza gdy do komputera podłączano dwa zewnętrzne ekrany po złączach cyfrowych (np. HDMI, DisplayPort, DVI). Finalnie problem rozwiązano wydajniejszym komputerem, jednakże pośredniom rozwiązaniem było użycie portu VGA, który jest mniej obciążający dla karty graficznej.

\subsubsection{Światło}
\paragraph{}
Kolejny prefabrykat stworzono by zachować spójność w oświetleniu trójwymiarowej sceny. Służy on do zgrupowania wszystkich źródeł wirtualnego światała. Jest to element bez zwwartej logiki biznesowej. Stworzono go w celu zachowania porządku w projekcie.


\subsection{SocketIO}
\paragraph{}
Jest to prefabrykat dostarczony jako kompotent implementacji Socket.io w bibliotece Asset Store. Jest on udostępniony na licencji Open Source.  Musi być on umiejscowiony w każdej scenie, która korzysta z połączenia sieciowego. Umiejscowienie jest dowolne, gdyż jest to prefabrykat abstrakcyjny (nie posiada graficznej reprezentacji w trójwymiarowym modelu). W prefabrykacie wywołano skrpyt SocketIOComponent, który odpowiada za inicjalizacje komunukacji sieciowej.

\subsubsection{Network}
\paragraph{}
Jest to kolejny prefabrykat abstrakcyjny, który służy do uruchomienia klasy Network odpowiadającej za implementacje metod służących do dwustronnej komunikacji.
Prefabrykat ten jest nierozerwalnie złączony z SocketIO, gdyż bezpośrendio korzysta z metod dostarczonych przez tą bibliotekę.


\begin{lstlisting}[language=CSharp]
private SocketIOComponent socket;

// Use this for initialization
void Start () {
	GameObject go = GameObject.Find("SocketIO");
	socket = go.GetComponent<SocketIOComponent>();

	socket.On("open", InitGame);
	socket.On("button", Button);
}
\end{lstlisting}
\captionof{lstlisting}{
	Inicjalizacja skryptu
}

\paragraph{}
Na początku wyszukiwaniy jest obiekt gry o określonej nazwie, a następnie pobierany komponent, czyli obiekt klasy. Warto pamiętać, ze połączenie inicjalizowane jest już przy uruchomieniu, więc nie ma potrzeby ,,ręcznego'' zestawiania warstwy sieciowej.

Metoda On w klase SocketIOComponent odpowiada za nasłuchiwanie serwera. Jako pierwszy parametr przymuje ciąg znaków określający nazwę metody. Natomiast drugi to referencja do metody, która wywoła się podczas wywołania akcji o nazwie wynikającej z pierwszego parametru.

Na potrzeby łatwiejszego zarządzania zdarzeniami i uniknięcia błędów w nazwach przycisków stworzono enum, zawierający nazwy obsługiwanych przycisków. Wszystkie metody odpowiadające za komunikację, a zwłąszcza za zarządzanie stanami oraz nazwami przycisków powinny przyjmować w parametrze opisywany atrybut wyliczalny.

\begin{lstlisting}[language=CSharp]
using System;

namespace AssemblyCSharp
{
	public enum ButtonEnum
	{
		BUTTON1, BUTTON2, BUTTON3, BUTTON4, BUTTON5, BUTTON6, BUTTON7, BUTTON8, BUTTON9, BUTTON10
	}
}

\end{lstlisting}
\captionof{lstlisting}{
	Definicja przycisków.
}

\paragraph{}
Metoda zarejestrowana pod nazwą ,,open'' wywoła się przy udanym zestawieniu połączenia. Jest to najlepsze miejsce do wstępnej konfiugracji planszy gry oraz przycisków w kontrolerze.

\begin{lstlisting}[language=CSharp]
public void InitGame(SocketIOEvent e)
{
	socket.Emit("register_game");
	EnableButton(ButtonEnum.BUTTON1);
	SetText(ButtonEnum.BUTTON1, "Nazwa 1");
	DisableButton(ButtonEnum.BUTTON2);
}
\end{lstlisting}
\captionof{lstlisting}{
	Inicjalizacja komponentów
}

\paragraph{}
Powyższy przykład inicjalizacji pokazuje emotowanie akcji do serwera o nazwie ,,register\_game''. Służy on do zarejestrowania gry na serwerze. Od tego czasu wszystkie akcje z kontrolera (np. wciśnięcie przycisku) będzie emitowane do gry.
Dodatkowo ukazano wstępną konfigurację przycisków: Uaktywnienie przycisku pierwszego, ustawienie okreśonej nazwy oraz wyłączenie przycisku drugiego.

\subsubsection{Budowanie zapytania}


{\color{red}Dokończyć opisywać logikę, network manager itd}.

\subsubsection{Replikator}
\paragraph{}
{\color{red}Opisać replikator}
\begin{center}
\includegraphics[width=0.7\textwidth]{images/replikator.png}
\captionof{figure}{
Konfiguracja replikatora
}
\end{center}

\begin{lstlisting}[language=CSharp]
...
public void Run () {
	StartCoroutine(Runner());

	NetworkManager.StartListening("button_left", Left);
	NetworkManager.StartListening("button_right", Right);
	NetworkManager.StartListening("button_kilof", Kilof);
	NetworkManager.StartListening("button_lopata", Lopata);
	NetworkManager.StartListening("button_jump", Jump);
	NetworkManager.StartListening("button_spadochron", Drabina);
	NetworkManager.StartListening("button_rotate", Rotate);
	NetworkManager.StartListening("button_startstop", Startstop);
	NetworkManager.StartListening("button_reset", Reset);
}

IEnumerator Runner() {
	while (lemmingCount < LemmingSize) {
		Create ();
		lemmingCount += 1;
		yield return new WaitForSeconds (secoundLimit);
	}
}
...
\end{lstlisting}
\captionof{lstlisting}{
	Uruchomienie replikatora
}

\begin{lstlisting}[language=CSharp]
...
void Left () {
	Lemming2.GetPrev ();
}

void Right () {
	Lemming2.GetNext ();
}

void Kilof () {
	if (Lemming2.activeEl) {
		Lemming2.activeEl.ToggleKilof ();
	}
}
...
\end{lstlisting}
\captionof{lstlisting}{
	Implementacja akcji
}


{\color{red}Opisać pozostałe prefabrykaty}

\subsection{Logika biznesowa}
\paragraph{}
{\color{red}Opisać}

\subsection{Serwer komunikacyjny}
\paragraph{}
{\color{red}Opisać}
\newpage
\section{Aplikacja mobilna - kontroler}

\begin{center}
\includegraphics[width=1\textwidth]{images/button_mockup.png}
\captionof{figure}{
Makieta - układ przycisków
}
\end{center}
\paragraph{}
Głównym założeniem było stworzenie uniwersalnego kontrolera przygotowanego pod dowolny rodzaj gry, bądź innej wizualizacji stworzonej w środowisku Unity. Podczas uruchomienia kontrolera serwer wysyła statusy przycisków oraz pól tekstowych.


\begin{center}
\includegraphics[width=1\textwidth]{images/graph1.png}

\includegraphics[width=1\textwidth]{images/graph2.png}
\captionof{figure}{
Przykładowa wizualizacja kontrolera
}
\end{center}

\subsection{Przyciski}
\paragraph{}
Każdy przycisk może zostać skonfigurowany poprzez ustawienie tekstu. Dodatkowo można zablokować przycisk podczas gdy nie jest on potrzebny w danym czasie. 
Jednym z główny założeń architektonicznych był rozdzielenie warstwy widoku od logiki biznesowej. Ustalono, że stworzenie nowego przycisku odbywać się będzie wymagało tylko dodania definicji w warstwie widoku (plik layout w formacie XML).

\begin{lstlisting}[language=XML]
<pl.pjatk.remotecontroller.CustomButton
    app:name="button2"
    android:layout_gravity="center_horizontal"
    android:layout_height="wrap_content"
    android:layout_width="wrap_content"
/>
\end{lstlisting}
\captionof{lstlisting}{
	Przykład zdefiniowanego przycisku
}
\paragraph{}
Na potrzeby realizacji powyższego założenia stworzono klasę CustomButton, która jest rozszerzeniem (dziedziczy bezpośrednio) klasy Button znajdującej się w pakiecie  android.widget.
\paragraph{}
Każdy z przycisków musi posiadać własną nazwę kodową, gdyż serwer podczas komunikacji sieciowej identyfikuje przycisk poprzez unikalny klucz. Domyślnie w środowisku Android każdy komponent wizualny może posiadać swoje Id, jednakże jest one reprezentowane poprzez liczbę typu Integer.
Dla ułatwienia dalszego rozwoju aplikacji postanowiono stworzyć nowy atrybut. Ich definicje umieszczas ię w formacie XML w pliku attrs.xml.

\begin{lstlisting}[language=XML]
<resources>
    <declare-styleable name="CustomButton">
        <attr name="name" format="string" />
    </declare-styleable>
</resources>
\end{lstlisting}
\captionof{lstlisting}{
    Definicja atrybutu o nazwie name
}

\paragraph{}
W konstruktorze poza domyślnymi wywołaniami klasy bazowej \texttt{Button} zapisywana jest wartość atrybutu name do zmiennej o tej nazwie oraz następuje wstępna konfiguracja przycisku.

\begin{lstlisting}[language=Java]
private static HashMap<String, CustomButton> buttons = new HashMap<String, CustomButton>();

 private void setUp() {
        buttons.put(getName(), this);
        setText(getName());

        setOnClickListener(new View.OnClickListener(){
            @Override
            public void onClick(View v) {
                ServerCommunication.click_button(getName());
            }
        });

    }
\end{lstlisting}
\captionof{lstlisting}{
    Konfiguracja w klasie CustomButton
}

\paragraph{}
Dzięki metodzie \texttt{setUp} konfigurowany jest przycisk. Ustawiana jest nazwa przycisku (zmiana możliwa za pomocą aplikacji Unity), przycisk jest zapisywany do listy wszystkich dostępych przycisków (dzięki temu konfiguracja ilośći przycisków odbywa się tylko w jednym miejscu - na poziomie warstwy widoku) oraz uruchamiany jest listener. Przy każdym kliknięciu wywułuje się metoda w singletonie \texttt{SerwerCommunication}.


\subsection{Użycie styli}
\paragraph{}
Dodatkowym wymaganiem było umożliwienie szybkiej zmiany wyglądu przycisków w trakcie działania aplikacji. Użycie natywnych styli niestety nie jest możliwe bez ponownego renderowania widoku. Aby zaoszczędzić czas i moc obliczeniową postanowiono, iż zostatnie użyta technologia zmiany tła za pomocą metody \texttt{backgroundResource}. Różne wyglądy przycisku mogą być definiowane jako selektory (zewnętrzne pliki xml w katalogu drawable). Ważne, by plik był odpowiednio nazwany, gdyż po nazwie następuje wyszukiwanie schematu podczas zmiany wyglądu.

\begin{lstlisting}[language=Xml]
<?xml version="1.0" encoding="utf-8"?>
<selector 
xmlns:android="http://schemas.android.com/apk/res/android
">
    <item android:state_pressed="true" >
        <shape>
            <gradient
                android:startColor="#bf1d00"
                android:endColor="#801300"
                android:angle="270" />
            <corners android:radius="10dp" />
            <stroke
                android:width="1dp"
                android:color="#71c2eb" />
        </shape>
    </item>
    <item>
        <shape 
xmlns:android="http://schemas.android.com/apk/res/android"
        android:shape="rectangle">
            <gradient android:startColor="#FFFFFF"
                android:endColor="#999"
                android:angle="270" />
        </shape>
    </item>
</selector>
\end{lstlisting}
\captionof{lstlisting}{
    Przykład zdefiniowanego tła
}
\paragraph{}
Wymaganiem była jednoczesna zmiana wyglądu wszystkich przycisków. Rozwiązano to za pomocą metody statycznej, która iteruje po wszystkich przyciskach i wywołuje metdę \texttt{backgroundResource}.


\begin{lstlisting}[language=Java]
 public static void setLayout(int i) {
    for (CustomButton btn : buttons.values()) {
        btn.setBackgroundResource(i);
    }
}
\end{lstlisting}
\captionof{lstlisting}{
   Zmiana tła dla wszystkich przcisków
}


\begin{lstlisting}[language=Java]
CustomButton.setLayout(R.drawable.dark);
\end{lstlisting}
\captionof{lstlisting}{
   Przykład wywołania zmiany tła przycisków
}

\subsection{Komunikacja sieciowa}

\begin{lstlisting}[language=Java]
public  class ServerCommunication {
    private static Socket mSocket = null;
    private static AppCompatActivity activity = null;

    public static void start(AppCompatActivity mainActivity) {
        activity = mainActivity;
        ServerCommunication.start();
    }

    private static void start() {
        if(mSocket == null) {
            Log.i("socket", "createServer");
            try {
                mSocket = IO.socket("http://192.168.0.12:5555");
                mSocket.connect();
                ServerCommunication.listenDisableButton();
                ServerCommunication.listenEnableButton();
                ServerCommunication.listenSetText();
                mSocket.emit("register_controller");
            } catch (URISyntaxException e) {
            }
        }
    }


    public static void click_button(String buttonName) {
        ServerCommunication.start();
        mSocket.emit("click", buttonName);
    }

    public static void listenDisableButton() {
        ServerCommunication.start();
        mSocket.on("disable_button", new Emitter.Listener() {
            @Override
            public void call(Object... args) {
                Log.i("socket", "disable_button");
                try {
                    JSONObject mainObject = new JSONObject(args[0].toString());
                    final String name = mainObject.getString("name");
                    activity.runOnUiThread(new Runnable() {
                       public void run() {
                           CustomButton.disableButton(name);
                       }
                   });
                } catch (JSONException e) {
                    e.printStackTrace();
                }
            }
        });
    }
...
\end{lstlisting}
\captionof{lstlisting}{
   Komunikacja sieciowa
}
\paragraph{}
Klasa \texttt{ServerCommunication} to prosta implementacja biblioteki Socket.IO.  Posiada ona metody statyczne, służące do nasłuchiwania akcji przychodzących z serwera lub emitowania komunikatów do aplikacji Unity. Każde wywołanie sprawdza na początku czy połączenie socket jest aktywne, jeśli nie, to tworzone jest nowe połączenie i uruchomiane domyślne nasłuchiwania na akcje. 
Ważne jest to, iż wszystkie nasłuchiwania tworzone są w nowych wątkach i wykonanie akcji aktualizacji graficznego interfejsu bezpośrednio nie jest możliwe. Należy wtedy utworzyć nowy obiekt klasy Runnable i uruchomić go za pomocą metody \texttt{activity.runOnUiThread}. Do obsługi bardziej złożonych komunikatów JSON używana jest standardowa klasa Javy - \texttt{JSONObject}.
\begin{lstlisting}[language=Java]
...
} catch (Exception e) {
    Context context = activity.getApplicationContext();
    CharSequence text = "Communication error";
    int duration = Toast.LENGTH_SHORT;

    Toast toast = Toast.makeText(context, text, duration);
    toast.show();
}
...
\end{lstlisting}
\captionof{lstlisting}{
   Przykład obsługi wyjątków
}
\paragraph{}
Jako graficzną reprezentację potencjalnych błędów (wyjątków w kodzie) można zastosować komunikaty pojawiające się poprzez zastosowanie klasy Toast dostępnej w Android SDK.

\section{Środowisko uruchomieniowe}
\paragraph{}
Powyżej opisana aplikacja została uruchomiona testowo w laboratorium na Polsko-Japońskiej Akademii Technik Komputerowych.

\subsection{Aplikacja główna - Unity}
\paragraph{}
Aplikacja została uruchomiona na komputerze przenośnym (laptop) posiadającym kartę graficzną umożliwiajacą podłączenie dwóch zewnętrznych ekranów - projektorów. Pierwszy z nich został połączony za pomocą złącza cyfrowego HDMI, natomiast drugi łączem DVI.
\subsection{Serwer komunikatów}
Serwer został uruchomiony na tym samym urządzeniu co aplikacja główna. Do uruchomienia nezbędne było zainstalowanie Node.JS wraz z menadżerem zależności - NPM.
\subsection{Aplikacja mobilna - kontroler}
\paragraph{}
Kontroler został uruchomiony na urządzeniu Xiaomi Mi4c wyposażonym w system Android w wersji 5.1. Urządzenie sprawdziło się jako kontroler, gdyż posiada ekran o rozmiarze 5 cali. Ilość pamięci operacyjnej była wystarczająca. Aplikacja zużywała tylko około 2-4\% pamięci RAM. 
\chapter{Inne implementacje}
\section{Google Cardboard}
\paragraph{}
Innym przykładem implementacji projektu w rozszerzonej rzeczywistości może być platforma Google Cardboard \cite{cardboard}. Są to niskobudżetowe okulary stworzone przez firmę Google do wyświetlania wirtualnej rzeczywistości. Kartonowe okulary powstały w celu wyświetlania obrazu stereoskopowego. Posiadają one miejsce do umieszczenia dowolnego telefonu komórkowego typu smartphone. Do projektu wybrano wersję, która pozwala na umieszczenie telefonu w sposób taki, iż tylna kamera nie jest zasłonięta przez obudowę. Dzięki temu można użyć platformę Cardboard przeznaczoną pierwotnie tylko do wirtualnej rzeczywistości do stworzenia aplikacji wykorzystującą augmented reality.
\begin{center}
\includegraphics[width=0.9\textwidth]{images/cardboard.jpg}
\captionof{figure}{
Google Cardboard w wersji do samodzielnego złożenia\cite{gizm}
}
\end{center}
Dzięki użyciu kamery użytkownik widzi obraz znajdujący się przed nim. Aby obraz był stereoskopowy należało stworzyć aplikację, która wyświetlić strumień danych z kamery dzieląc go na dwa obrazy (kolejno dla lewego oraz prawego oka).

\begin{center}
\includegraphics[width=0.9\textwidth]{images/kadr.jpg}
\captionof{figure}{
Ujęcie z kamery aparatu w obrazie stereoskopowym - scena w Unity
}
\end{center}

\subsection{ArToolkit}
\paragraph{}
W celu wyświetlenia wirtualnych obiektów na obrazie z kamery rozbudowano aplikację z porzednich rozdziałów. Do platformy Unity doinstalowano zewnętrzny komponent ARToolKit\cite{ar}. Jest to biblioteka wydana przez University of Washington\cite{ar2}, lecz obecnie upostępniona jest na licencji GNU. Kod źródłowy jest otwarty i rozwijany przez środowisko Open Source\cite{ar3}.
\subsection{Markery}
\paragraph{}
Za pomocą tej biblioteki możliwe jest wykrywanie w obrazie z kamery markerów, czyli specjalnie przygotowanych czarno-białych obrazków (w naiwnej implementacji - wydrukowanych na kartkach), oraz nakładanie w ich miejsce trójwymiarowych modeli lub całych scen. Dzięki bibliotece ArToolkit możliwe jest diagnozowanie pod jakim kątem padania oraz w jakiej odległości od urządzenia znajduje się marker. Umiejscowienie tagu analizowane jest w czasie rzeczywistym, co zapewni ciągłą korekcję ułożenia wirtualnych modeli względem ich realnych odpowiedników.

\begin{center}
\includegraphics[width=0.5\textwidth]{images/hiro.png}
\captionof{figure}{
Przykład przygotowanego obrazka do rozpoznawania - marker
}

\end{center}
\paragraph{}
Szablony markerów można wykonywać we własnym zakresie. Aby zaimportować nowe obrazki do biblioteki ArToolkit należy przygotować specjalny plik binarny reprezentujący model markera\cite{marker}.

\begin{center}
\includegraphics[width=1\textwidth]{images/artoolkit-demo.jpg}
\captionof{figure}{
Przykład zastosowania markera w ARToolkit\cite{cardboard2}
}
\end{center}
\subsection{Przykład implementacji}
\begin{center}
\includegraphics[width=1\textwidth]{images/artoolkit-przyklad.png}
\captionof{figure}{
Podstawowa impementacja
}
\end{center}
\paragraph{}
Implementacja na platformie Unity przy użyciu ArToolkit jest zasadniczo prosta. Przy użyciu wcześniej przygotowanych kodów markerów (domyślnie dostępne są dwa) i zastosowaniu prefabrykatów w trójwymiarowej przestrzeni można wskazać miejsce od którego będzie odliczana odległość do pozostałych elementów. np. Jeżeli bryła ustawiona zostanie w odległości 5cm od prefabrykatu to podczas analizy obrazu i wykryciu rzeczywistego markera to bryła zostanie umiejscowiona na ekranie w tej samej odległości.
\subsection{Napotkane problemy i ograniczenia}
\paragraph{}
\begin{enumerate}
	\item Proces renderowania musi odbywać się na telefonie komórkowym, gdyż obraz kamery jest ciągle analizowany. Przy dużych scenach stworzonych w Unity moc obliczeniowa urządzenia jest niewystarczająca.
	\item Analizowanie pozycji markera przy nieustannie włączonej kamerze powoduje dużą drenację baterii urządzenia. Czas pracy na baterii jest mocno ograniczony
	\item Unity w wersji Personal (darmowej) skompilowanej pod platrformę mobilną (np. Android) nie udostępnia obsługi sieci (np. za pomocą połączenia TCP). Nie pozwala to na połączenie z zenętrznym kontrolerem.
	\item Sterowanie za pomocą kontrolera bez fizycznych przycisków z założonymi okularami Google Cardboard jest bardzo uciążliwe. W przyszłości należałoby rozważyć połączenie telefou z zewnętrznym kontrolerem typu PAD\cite{pad}.
\end{enumerate}
\newpage
\section{Inne implementacje - Project Tango}
\paragraph{}
Kolejnym przykładem implementacji może być platforma Google Project Tango\cite{tango}. Jest to platforma rozszerzonej rzeczywistości zapoczątkowana przez Johny'ego Lee (współtwórcy między innymi Microsoft Kinect\cite{autor}) w 2014. 
\paragraph{}
Idea projektu jest bardzo podobna jak przykład zaprezentowany w poprzednim rozdziale. Jednakże Project Tango to również podzespoły sprzętowe. Twórcy zastosowali specjalne kamery do pomiaru głębi oraz analizy ruchu (technologia podczerwieni). Kamery te korzystają z technologii Intel Real Sense. Dzięki temu urządzenie potrafi analizować obraz kamery i mapować go na trójwymiarowy obraz. Z dokładnością do milimetra urządzenie jest w stanie określić wymiary realnych elementów znajdujących się przed kamerą. Dzięki temu nie ma potrzeb używania zbędnych fizycznych markerów do określenia miejsca w którym znajduje się odbiorca z urządzeniem.
\paragraph{}
Firma Google zaprezentowała projekt w 2014 roku wraz z dwoma urządzeniami testowymi (The Yellowstone tablet,  The Peanut phone). Jednakże te urządzenia nie trafiły nigdy na rynek komercyjny. Dopiero w 2016 roku firma Lenovo zaprezentowała pierwszy masowo produkowany telefon obsługujący Project Tango - Lenovo Phab2 Pro.
\paragraph{}
Projekt pod początku udostępnia developerom możliwość tworzenie aplikacji za pomocą API do języków Java oraz C. Dodatkowo udostępniona jest SDK (Software Development Kit) wraz z obszerną dokumentacją do platformy Unity\cite{tangounity}.
\paragraph{}
Jedynym środowiskiem uruchomieniowym dostępnym na obecną chwilę jest Android, chociaż Google zapowiada możliwość w przyszłości uruchomienia w środowisku Windows 10.

\begin{center}
\includegraphics[width=0.9\textwidth]{images/tango.jpg}
\captionof{figure}{
Prototyp urządzenia
}
\end{center}

\paragraph{}
Google zaprezentowało również okulary do wirtualnej rzeczywistości. Jednakże był to tylko prototyp. Obecnie na rynku nie ma urządzenia dostosowanego do tego celu. Jedyną możliwością byłoby użycie Lenovo Phab Pro wraz z Google Cardboard.

\subsection{Wady i zalety}
\paragraph{}
\begin{enumerate}
	\item Project Tango jest stworzony jako kooperacja dedykowanego sprzętu oraz specjalistycznego oprogramowania, co za tym idzie wydajność urządzeń powinna być duża.
	\item Dużą wadą jest to, iż na rynku dopiero pojawił się pierwszy smartphone z obsługą projektu. Technologia wydaje się być na początkowej fazie rozwoju.
	\item Dzięki tej technologii można zrezygnować z markerów opisanych w poprzednim rozdziale. Odczucia użytkowników powinny być bardziej intuicyjne.
\end{enumerate}

\newpage
\section{Wnioski i dalszy rozwój}
\paragraph{}
Obserwując rynek nowych technologii można zauważyć duże zainteresowanie zarówno rozszerzoną jak i wirtualną rzeczywistością. Producenci eksperymentują z różnymi urządzeniami, jednakże większość ich jest w fazie testów i nie jest dostępna dla potencjalnych konsumentów.
Wiele z firm próbuje tworzyć własne standardy, jednakże żaden 


\paragraph{}
{\color{red}Napisać o tym, ze Microsoft otworzyl platforme hololens, wiec w przyszłosci będą nowe urządzenia}




%\renewcommand{\publ}{}
\cleardoublepage
\small
\listoffigures
\lstlistoflistings


\begin{thebibliography}{2}
\bibitem{wea} https://pl.wikipedia.org/wiki/Wearables
\bibitem{boeing} http://www.komputerswiat.pl/nowosci/wydarzenia/2012/28/boeing-z-androidem-na-pokladzie.aspx.
\bibitem{nui} https://en.wikipedia.org/wiki/Natural\_user\_interface
\bibitem{kinect} http://www.xbox.com/pl-PL/xbox-one/accessories/kinect-for-xbox-one
\bibitem{leap} https://www.leapmotion.com/
\bibitem{emotiv} http://emotiv.com
\bibitem{cinema} http://cinema-city.pl/4dx-info
\bibitem{kr} http://www.techsty.art.pl/hipertekst/cyberprzestrzen/krueger.htm
\bibitem{glow} http://dada.compart-bremen.de/item/artwork/1347
\bibitem{metaplay} http://dada.compart-bremen.de/item/artwork/1348
\bibitem{videoplace} http://dada.compart-bremen.de/item/artwork/1346
\bibitem{holo2} https://www.microsoft.com/microsoft-hololens/en-us/why-hololens
\bibitem{lemmings} https://en.wikipedia.org/wiki/Lemmings
\bibitem{unity1} https://store.unity3d.com/subscribe
\bibitem{unity2} https://unity3d.com/unity/multiplatform
\bibitem{market} https://www.netmarketshare.com/operating-system-market-share.aspx?qprid=10
\bibitem{cardboard2} http://arblog.inglobetechnologies.com/?p=421
\bibitem{socket} http://socket.io/
\bibitem{prefab} http://docs.unity3d.com/Manual/Prefabs.html
\bibitem{cardboard} https://vr.google.com/cardboard/index.html
\bibitem{gizm} http://gizmodo.com/turn-your-android-into-a-virtual-reality-headset-with-g-1596026538
\bibitem{ar} http://artoolkit.org/
\bibitem{ar2} https://www.hitl.washington.edu/artoolkit/
\bibitem{ar3} https://github.com/artoolkit
\bibitem{marker} http://bit.ly/1YU199f
\bibitem{pad} https://pl.wikipedia.org/wiki/Gamepad
\bibitem{tango} https://get.google.com/tango/
\bibitem{autor} https://en.wikipedia.org/wiki/Johnny\_Lee\_(computer\_scientist)
\bibitem{tangounity} https://developers.google.com/tango/apis/unity/
\bibitem{holo} https://www.engadget.com/2016/06/01/microsoft-opens-the-hololens-platform-to-other-companies/
\bibitem{pa} Building Microservices,  Sam Newman , Wydanie 4, 2016
\end{thebibliography}

%\cleardoublepage
%\addcontentsline{toc}{chapter}{Index}
%\printindex
    
\end{document}